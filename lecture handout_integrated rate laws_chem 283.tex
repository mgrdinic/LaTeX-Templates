%%%%%%%%%%%%%%%%%%%%%%%%%%%%%%%%%%%%%%%%%%%%%%%%%%%%
%												   %
%                 Notebook Insert                  %
%												   %
%%%%%%%%%%%%%%%%%%%%%%%%%%%%%%%%%%%%%%%%%%%%%%%%%%%%         


\documentclass[11pt, letter]{article}
%\usepackage{geometry}
\usepackage[inner=1.0cm, outer=4.5cm, top=2.5cm, bottom=2.5cm]{geometry}
\pagestyle{empty}

\usepackage{graphicx}
\usepackage{nopageno}
\usepackage{mhchem}
\usepackage{booktabs}



%%%%%%%%%%%%%%%%%%%%%%%%%%%%%%%%%%%%


\begin{document}




\noindent\textsc{\textbf{Integrated Rate Laws, Part 1}}

\vspace{0.5cm}

\noindent Determining the rate law using initial reaction rate data is time consuming and often difficult (perhaps even impossible) if we can't accurately measure the rate at the instant the reaction begins. It would be ideal if the rate law could be determined by one single experiment.

\vspace{.2cm}

\noindent This can be done for reactions in which the rate depends on the concentration of only one substance. One good example of this is the photochemical decomposition of ozone in the stratosphere:

\vspace{.2cm} 

\begin{equation}
  \ce{O3(g) ->[sunlight] O2(g) + O(g)}
\end{equation}

\vspace{.2cm} 

\noindent One experiment gave the results in the table below. Plot this data set.


\vspace{.5cm} 

\begin{tabular}{l c}
\toprule
\textbf{Time (s)} & {\ce{[O3] \,(M) 10^-4} \\
\midrule
 0.0 & 1.000 \\
 100.0 & 0.896 \\
 200.0 & 0.803 \\
 300 & 0.719 \\
 400.0 & 0.6444 \\
 500.0 & 0.577 \\
 600.0 & 0.517 \\
\bottomrule
\end{tabular}
 
\vspace{1cm}

\noindent\textsc{\textbf{Integrated Rate Laws}}

\vspace{0.5cm}

\noindent Determining the rate law using initial reaction rate data is time consuming and often difficult (perhaps even impossible) if we can't accurately measure the rate at the instant the reaction begins. It would be ideal if the rate law could be determined by one single experiment.

\vspace{.2cm}

\noindent This can be done for reactions in which the rate depends on the concentration of only one substance. One good example of this is the photochemical decomposition of ozone in the stratosphere:

\vspace{.2cm} 

\begin{equation}
  \ce{O3(g) ->[sunlight] O2(g) + O(g)}
\end{equation}

\vspace{.2cm} 

\noindent One experiment gave the results in the table below. Plot this data set.


\vspace{.5cm} 

\begin{tabular}{l c}
\toprule
\textbf{Time (s)} & {\ce{[O3] \,(M) 10^-4} \\
\midrule
 0.0 & 1.000 \\
 100.0 & 0.896 \\
 200.0 & 0.803 \\
 300 & 0.719 \\
 400.0 & 0.6444 \\
 500.0 & 0.577 \\
 600.0 & 0.517 \\
\bottomrule
\end{tabular}
  


 


% THE END 

\end{document} 
